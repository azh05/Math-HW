\documentclass[12pt]{article}

\usepackage{graphicx}			% Use this package to include images
\usepackage{amsmath}			% A library of many standard math expressions
\usepackage{amssymb}
\usepackage[margin=1in]{geometry}% Sets 1in margins. 
\usepackage{fancyhdr}			% Creates headers and footers
\usepackage{enumerate}          %These two package give custom labels to a list
\usepackage[shortlabels]{enumitem}


% Creates the header and footer. You can adjust the look and feel of these here.
\pagestyle{fancy}
\fancyhead[l]{Anthony Zhao}
\fancyhead[c]{Math 131AH Homework \#1}
\fancyhead[r]{\today}
\fancyfoot[c]{\thepage}
\renewcommand{\headrulewidth}{0.2pt} %Creates a horizontal line underneath the header
\setlength{\headheight}{15pt} %Sets enough space for the header



\begin{document} %The writing for your homework should all come after this. 
%FromSection2.4: 4*,6,15*,19,23,36*. FromSection1.1: 3,5,7,12*,17,34*.

%Enumerate starts a list of problems so you can put each homework problem after each item. 
\begin{enumerate}[start=1,label={\bfseries Exercise \arabic*:},leftmargin=1in] %You can change "Problem" to be whatever label you like. 
    \item \begin{enumerate}
        \item You want to live a happy life and tie it to people and things, not a goal
        \item The plane leaves and you are not on it, and you do not regret it.
        \item There is someone for which there isn't someone that makes them unhappy.
    \end{enumerate}
    
    \item \textbf{Base Case:}
    \[
    2(1) - 1 = 1 = 1^2
    \]
    \textbf{Induction Hypothesis:} Assume for some $n \in \mathbb{N}$ that $1 + \cdots + (2n - 1) = n^2$. 
    
    \textbf{Induction Step:} We want to show that it holds for $n + 1$. 

    Expanding the left hand side (line 1) and using the induction hypothesis (line 3),
    \begin{align*}
        1 + \cdots + (2(n+1) - 1) &= 1 + \cdots + (2n - 1) + (2(n+1) - 1)\\
        &= 1 + \cdots + (2n - 1) + (2n + 2 - 1) \\
        &= n^2 + 2n + 1\\
        &= (n + 1)^2
    \end{align*}

    Hence, the statement holds for $n+1$. 

    \item \textbf{Base Case:} 
    \begin{align*}
        1^3 + 2^3 + 3^3 &= 1 + 8 + 27\\
        &= 36
    \end{align*}
    36 is divisble by 9

    \textbf{Induction Hypothesis:} Assume for some $n \in \mathbb{N}$ that $(n)^3 + (n+1)^3 + (n+2)^3$ is divisible by 9. 
    So, $(n)^3 + (n+1)^3 + (n+2)^3 = 9k$ for some $k \in \mathbb{N}$.

    \textbf{Induction Step:} We want to show that it holds for $n+1$. 

    Expanding the term and using the induction hypothesis, we get the following
    \begin{align*}
        (n+1)^3 + (n+2)^3 + (n+3)^3 &= (n+1)^3 + (n+2)^3 + n^3 + 3n^2 + 9n + 27 \\
        &= n^3 + (n+1)^3 + (n+2)^3 + 9(n + 27) \\ 
        &= 9k + 9(n + 27) \\
        &= 9(k + n + 27)
    \end{align*}
    Since $k, n, 27 \in \mathbb{N}$, the sum is also divisible by 9.
    \item \textbf{Base Case:}
    \begin{align*}
        1^5 - 1 = 0
    \end{align*}
    0 is divisible by 30. 

    \textbf{Induction Hypothesis:} Assume for some $n \in \mathbb{N}$ that $n^5 - n$ is divisible by 5.
    That is $n^5 - n = 30k$ for some $k \in \mathbb{N}$.

    \textbf{Induction Step:} We want to show that it holds for $n + 1$.
    
    Expanding the $(n+1)^5$ term and applying the induction hypothesis, we get the following.
    \begin{align*}
        (n+1)^5 - (n+1) &= n^5 + 5n^4 + 10n^3 + 10n^2 + 5n + 1 - n - 1 \\
        &= n^5 - n + 5n^4 + 10n^3 + 10n^2 + 5n \\
        &= 30k + 5(n^4 + 2n^3 + 2n^2 + n)\\
        &= 30k + 5n(n+1)(n^2+n+1) \\
        &= 30k + 5n(n+1)(n^2 + 4n + 4 - 3n - 3)\\
        &= 30k + 5n(n+1)(n+2)^2 - 15n(n+1)
    \end{align*} 
    
    We know that $n(n+1)$ must be even because either $n$ or $n+1$ is even, so the third term is divisible by 30.
    We also know that $n(n+1)(n+2)$ must be divisible by 6 because at least one of the terms in the product must be divisible by 3 and one of the terms in the product must be divisible by 2. 
    Therefore, it holds for $n+1$

    \item We want to show that 
    \[
        \frac{1^2}{1\cdot 3} + \cdots + \frac{n^2}{(2n-1)(2n+1)} = \frac{\sum^n_{i=1}i}{2n + 1}
    \]
    for all natural numbers $n \geq 1$.

    \textbf{Base Case:} 
    \[
    \frac{1}{1\cdot3} = \frac{1}{3} = \frac{2(1)-1}{2(1)+1}
    \]

    \textbf{Induction Hypothesis:}  Assume for some $n \in \mathbb{N}$ that 
    \[
        \frac{1^2}{1\cdot 3} + \cdots + \frac{n^2}{(2n-1)(2n+1)} = \frac{\sum^n_{i=1}i}{2n + 1}
    \]
    \textbf{Induction Step:} We want to show that it holds for $n+1$. 

    Note that $\sum_{i=1}^n i = \frac{n(n+1)}{2}$.
    Expanding the sum and using the induction hypothesis,
    \begin{align*}
        \frac{1^2}{1\cdot 3} + \cdots + \frac{n^2}{(2n-1)(2n+1)} + \frac{(n+1)^2}{(2n+1)(2n+3)} &= \frac{\sum^n_{i=1}i}{2n + 1} +  \frac{(n+1)^2}{(2n+1)(2n+3)}\\
        &= \frac{n(n+1)}{2(2n+1)} + \frac{(n+1)^2}{(2n+1)(2n+3)}\\
        &= \frac{n(n+1)(2n+1) + 2(n+1)^2}{2(2n+1)(2n+3)}\\
        &= \frac{(n+1)(n+3)}{2(2n+3)}\\
        &= \frac{\sum^{n+1}_{i=1}i}{2(n+1)+1}
    \end{align*}

    Therefore, it holds for $n+1$. 
    \item Let $a, b \in \mathbb{Z}$ where $GCD(a, b) = 1$. Assume that $(\frac{a}{b})^2 = 6$. 
    So, $a^2 = 6b^2$. So, $a^2$ is divisible by 6, meaning that $a$ also must be divisible by 6, so
    $a^2 = 6k$ for some $k \in \mathbb{Z}$. Thus, 
    \begin{align*}
        a^2 &= (6k)^2 \\
        &= 36k^2\\
        36k^2 &= 6b^2 \\
        6k^2 &= b^2 \\
    \end{align*}
    This implies that $b^2$ and $b$ are divisible by 6. So, $a$ and $b$ are both divisible by 6, 
    which contradicts our claim that $GCD(a, b) = 1$. Therefore there cannot exist a rational number whose square is 6.
    \item $b$
    \item $b$
    \item $e$
    \item $g$
    \item $b$
\end{enumerate}

\end{document}
