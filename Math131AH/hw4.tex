\documentclass[12pt]{article}

\usepackage{graphicx}			% Use this package to include images
\usepackage{amsmath}			% A library of many standard math expressions
\usepackage{amssymb}
\usepackage[margin=1in]{geometry}% Sets 1in margins. 
\usepackage{fancyhdr}			% Creates headers and footers
\usepackage{enumerate}          %These two package give custom labels to a list
\usepackage[shortlabels]{enumitem}


% Creates the header and footer. You can adjust the look and feel of these here.
\pagestyle{fancy}
\fancyhead[l]{Anthony Zhao}
\fancyhead[c]{Math 131AH Homework \#4}
\fancyhead[r]{\today}
\fancyfoot[c]{\thepage}
\renewcommand{\headrulewidth}{0.2pt} %Creates a horizontal line underneath the header
\setlength{\headheight}{15pt} %Sets enough space for the header



\begin{document} %The writing for your homework should all come after this. 
%FromSection2.4: 4*,6,15*,19,23,36*. FromSection1.1: 3,5,7,12*,17,34*.

%Enumerate starts a list of problems so you can put each homework problem after each item. 
\begin{enumerate}[start=1,label={\bfseries Problem \arabic*:},leftmargin=1in] %You can change "Problem" to be whatever label you like. 
    \item $(\Rightarrow)$ Assume that $\{ a_{n} \}_{n \in \mathbb{N}}$ converges to $a$. We want to show that $\{ |a_{n}| \}_{n \in \mathbb{N}}$ converges to $|a|$.
    
    Let $\epsilon > 0$. Since $\{ a_{n} \}_{n \in \mathbb{N}}$ converges to $a$, there exists an $n_{\epsilon} \in \mathbb{N}$ such that for all $n \geq n_{\epsilon}$, $|a_{n} - a| < \epsilon$.
    We know that $||a_{n}| - |a|| \leq |a_{n} - a|$. Thus, $||a_{n}| - |a|| < \epsilon$ for all $n \geq n_{\epsilon}$. Hence, $\{ |a_{n}| \}_{n \in \mathbb{N}}$ converges to $|a|$.
    
    $(\nLeftarrow)$ Let $a_{n} = -1$ for all $n \in \mathbb{N}$. Evidentally, $\{|a_{n}|\}$ converges to $1$ but $\{a_{n}\}$ converges to $-1$. Hence, the converse is not true.

    \item Let $\epsilon > 0$. Note that $a_{n} = 1 + \sum^{n-1}_{i=1} \frac{1}{3^{i}}$. 
    
    Using some facts about sums and geometric series, we know that 
    \begin{align*}
        \sum^{\infty}_{i=1} ar^{n-1} &= \frac{a}{1-r}\\
        \sum^{n}_{i=1} ar^{n-1} &= \frac{a(1-r^{n})}{1-r}
    \end{align*}

    Subtracting the two and plugging in $a = 1$ and $r = \frac{1}{3}$, we get
    \begin{align*}
        \sum^{\infty}_{i=n} \frac{1}{3^{i}} = \frac{1}{1-\frac{1}{3}} - \frac{1-\frac{1}{3^{n}}}{1-\frac{1}{3}} = \frac{1-(1-\frac{1}{3^{n}})}{\frac{2}{3}} = \frac{3}{2}\frac{1}{3^{n}}
    \end{align*}
    Note the limit of the series is $ \sum^{\infty}_{i=0} \frac{1}{3^{i}} = \frac{3}{2}$, and the difference from that value from $a_{n}$ is $\frac{3}{2}\frac{1}{3^{n}}$. Thus, we want to choose a $n_{\epsilon}$ such that $\frac{3}{2}\frac{1}{3^{n_{\epsilon}}} < \epsilon$.

    Solving for $n_{\epsilon}$, we get $n_{\epsilon} > \log_{3}(\frac{3}{2\epsilon})$. Hence, for all $n \geq n_{\epsilon}$, $|a_{n} - \frac{3}{2}| < \epsilon$. Thus, $\{a_{n}\}$ converges to $\frac{3}{2}$.

    \item Since $\{ a_{n} \}$ is bounded, we know there exists some $M$ such that $a_{n} \leq M$ for all $n \in \mathbb{N}$. 
    
    Hence, $|a_{n}b_{n}| = |a_{n}||b_{n}| \leq |M||b_{n}|$. 

    Let $\epsilon > 0$. Since $\lim_{n \to \infty} b_{n} = 0$, we can find a $n_{\epsilon} \in \mathbb{N}$ such that $|b_{n} - 0| = |b_{n}| < \frac{\epsilon}{|M|}$. Note that $\frac{\epsilon}{|M|} > 0$. 

    Hence, for all $n \geq n_{\epsilon}$, $|a_{n}b_{n} - 0| = |a_{n}b_{n}| \leq |M||b_{n}| < |M|\frac{\epsilon}{|M|} = \epsilon$. Thus, $\lim_{n \to \infty} a_{n}b_{n} = 0$.

    \item Since we know $\lim_{n \to \infty} a_{n} = \lim_{n \to \infty} c_{n}$ then $\lim_{n \to \infty}(a_{n} - c_{n}) = 0$.
    
    Let $\epsilon > 0$. Since $\lim_{n \to \infty} b_{n} = 0$, we can find a $n_{\epsilon} \in \mathbb{N}$ such that $\left|  a_{n} - c_{n} \right| < \epsilon$. 
    Note that since we know $c_{n} \geq a_{n}$ for all $n \in \mathbb{N}$, $\left| a_{n} - c_{n} \right| = c_{n} - a_{n}$. 

    Using the fact that $a_{n} \leq b_{n}$, we know that $c_{n} - a_{n} \geq c_{n} - b_{n}$ for all $n \in \mathbb{N}$. 

    Hence, $|c_{n} - b_{n}| \leq |c_{n} - a_{n}|$ for all $n \in \mathbb{N}$. Thus, for all $n \geq n_{\epsilon}$, $|c_{n} - b_{n}| \geq |c_{n} - a_{n}| < \epsilon$. 
    So, $\lim_{n \to \infty} (c_{n} - b_{n}) = 0$. So, $\lim_{n \to \infty} b_{n} = \lim_{n \to \infty} c_{n} = \lim_{n \to \infty} a_{n}$.

    \item \begin{align*}
        \lim_{n \to \infty} \sqrt{4n^{2} + n} - 2n &= \lim_{n \to \infty} (\sqrt{4n^{2} + n} - 2n)\cdot \frac{\sqrt{4n^{2} + n} + 2n}{\sqrt{4n^{2} + n} + 2n}\\
        &= \lim_{n \to \infty} \frac{4n^{2} + n - 4n^{2}}{\sqrt{4n^{2} + n} + 2n}\\
        &= \lim_{n \to \infty} \frac{n}{\sqrt{4n^{2} + n} + 2n}\\
        &= \lim_{n \to \infty} \frac{n}{n(\sqrt{4 + \frac{1}{n}} + 2)}\\
        &= \frac{1}{\sqrt{4 + \lim_{n \to \infty}\frac{1}{n}} + 2}\\
        &= \frac{1}{4}
    \end{align*}
    \item \begin{enumerate}
        \item Assume that for all but finitely many $a_{n}$ we have $a_{n} \geq a$. Assume that $\lim_{n \to \infty} a_{n} = L < a$. 
        Let $\epsilon = a - L > 0$. We know that there exists an $n_{\epsilon} \in \mathbb{N}$ such that $|a_{n} - a| < \epsilon$ for all $n \geq n_{\epsilon}$.
        So, \begin{align*}
            -\epsilon < a_{n} - L &< \epsilon\\ 
            L - a < a_{n} - L &< a - L \\ 
            a_{n} &< a
        \end{align*} 
        Hence, for all $n \geq n_{\epsilon}$, $a_{n} < a$. So, there an infinite number of $a_{n} < a$. This is a contradiction. So, $\lim_{n \to \infty} a_{n} \geq a$.

        \item This can be shown similarly by contradiction. 
        \item We know that finitely many $a_{n}$ belong to the intervval $[ a, b]$. So, we know that for finitely many $a_{n} \geq a$ and $a_{n} \geq b$. Using the previous two parts,
        we know that $\lim_{n \to \infty} a_{n} \geq a$ and $\lim_{n \to \infty} a_{n} \leq b$. Hence, $a \leq \lim_{n \to \infty} a_{n} \leq b$, so $\lim_{n \to \infty} a_{n} \in [a, b]$.
    \end{enumerate}
    \item Let $\lim_{n \to \infty} a_{n} = L > a$. Let $\epsilon = L - a > 0$. 
    We know there exists a $n_{\epsilon}$ such that $|a_{n} - L| < \epsilon$ for all $n \geq n_{\epsilon}$.

    So, $\left|  a_{n} - L \right| < L - a$. 
    The case where $a_{n} > L$ is trivial, so we consider the case where $a_{n} < L$. 
    $\left| a_{n} - L \right| = L - a_{n}$. 
    So, $-a_{n} < -a$ which means $a_{n} > a$ for all $n \geq n_{\epsilon}$. Hence, we have found such $n_{\epsilon}$.

    \item Since $\{a_{n}\}$ is bounded, we know there exists some $M$ such that $a_{n} \leq M$ for all $n \in \mathbb{N}$.
    
    Let $\epsilon > 0$. Choose a $n_{\epsilon}$ such that $|a_{n} - a_{m}| < \frac{\epsilon}{M^{2}}$ for all $n, m \geq n_{\epsilon}$.
    So, $|a_{n}^2 - a_{m}^2| = |a_{n} - a_{m}||a_{n} + a_{m}| < \frac{\epsilon}{M^{2}}(2M) = \epsilon$ for all $n, m \geq n_{\epsilon}$. Hence, $\{a_{n}^{2}\}$ is Cauchy.

    \item 
    \begin{enumerate}
        \item Inductively, we can show that since $a_{1} = 3$ and $a_{n+1} = \frac{1}{2}a_{n} + \frac{2}{a_{n}}$, that if $a_{n}$ is a positive rational number, then $a_{n+1}$ must be one as well. 
        This is beacuse $\mathbb{Q}$ is closed under addition and multiplciation, and both terms in the reccurence must be positive.

        Hence, since $a_{n} > 0$ for all $n \in \mathbb{N}$, we can conclude that 0 is a lower bound for the sequence.
        \item By the above, we know that each $a_{n} \in \mathbb{Q}$. 
        \item We want to show that $\{a_{n}\}$ is monotonically decreasing. 
        
        So, we know that 
        \[
            a_{n+1} = \frac{a_{n}}{2} + \frac{1}{a_{n}} = \frac{a_{n}^{2} + 2}{2a_{n}}
        \]
        We want to show that $a_{n} \geq a_{n+1}$. 

        \begin{align*}
            a_{n} &\geq a_{n+1}\\ 
            a_{n} &\geq \frac{a_{n}^{2} + 2}{2a_{n}}\\
            a_{n}^{2} &\geq 2
        \end{align*}
        
        Using induction on $n$, we can show that $a_{n}^{2} \geq 2$ for all $n \in \mathbb{N}$. 

        Hence, we have shown that $a_{n} \geq a_{n+1}$ for all $n \in \mathbb{N}$, so $\{ a_{n} \}$ is monotonically decreasing.

        \item Since $\{ a_{n} \}$ is monotonically decreasing and bounded below, we know that it must converge. 
       
        Since $\lim_{n \to \infty} a_{n} = \lim_{n \to \infty} a_{n+1} = a$, we can plug this into the reccurence to find that $a = \frac{a^2+2}{2a}$. Solving for $a$, 
        we get $2a^{2} - a^{2} -2 = 0$. So, $a^{2} - 2 = 0$. Hence, $a = \pm \sqrt{2}$. However, we know that $a>0$, so $a = \sqrt{2}$.
    \end{enumerate}
    \item \begin{enumerate}
        \item By induction, we can show that for all $n \in \mathbb{N}$, $a_{n} < 2$. 
        
        \textbf{Base Case:} $a_{1} = 2 < 2$.

        \textbf{Induction Hypothesis:} Assume that $a_{n} < 2$ for some $n \in \mathbb{N}$.

        \textbf{Induction Step:} We want to show that $a_{n+1} < 2$. 
        We know that $2 + a_{n} < 4$ by the induction hypothesis. So, $\sqrt{2 + a_{n}} < 2$. 
        Hence, $a_{n+1} = \sqrt{2 + a_{n}} < 2$.

        Thus, there exists an upper bound for $a_{n}$ and it is 2.
        \item We want to show that $\{a_{n}\}$ is monotonically increasing. 
        \begin{align*}
            a_{n+1} = \sqrt{2 + a_{n}} &\geq a_{n}\\
            2 + a_{n} &\geq a_{n}^{2}\\
            a_{n}^{2} - a_{n} - 2 &\leq 0\\
            (a_{n} - 2)(a_{n} + 1) &\leq 0\\
            -1 \leq a_{n} \leq 2
        \end{align*}
        Evidentally, $a_{n}$ is greater than 0, since we can prove with induction that $a_{n} + 2 > 0$ for all $n \in \mathbb{N}$,
        and by part 1, we know that $2$ is an upper bound for $a_{n}$. Hence, $a_{n}$ is monotonically increasing because the inequality holds for all $n$. 

        \item Since $\{a_{n}\}$ is monotonically increasing and bounded above, we know that it must converge.
        Let $a = \lim_{n \to \infty} a_{n} = \lim_{n \to \infty} a_{n+1}$. Then, $a = \sqrt{2 + a}$. Solving for $a$, we get $a^{2} - a - 2 = 0$. Hence, $a = 2, -1$. Since $a > 0$, we know that $a = 2$.
    \end{enumerate}
    \item 
    \begin{enumerate}
        \item We want to show that $\{a_{n}\}$ is monotonically increasing.
    
        \textbf{Lemma:} $a_{n} \leq b_{n}$ for all $n \in \mathbb{N}$.
        
        \textbf{Base Case:} $0 < a_{1} < b_{1}$ is given. 
    
        \textbf{Induction Hypothesis:} Assume that $a_{n} \leq b_{n}$ for some $n \in \mathbb{N}$.
    
        \textbf{Induction Step:} We want to show that $a_{n+1} \leq b_{n+1}$.
    
        We know that $a_{n+1} = \sqrt{a_{n}b_{n}}$ and $b_{n+1} = \frac{a_{n} + b_{n}}{2}$.
        We can use induction to show that $a_{n} > 0$ and $b_{n} > 0$.  
        By the AM-GM inequality, we know that $\sqrt{a_{n}b_{n}} \leq \frac{a_{n} + b_{n}}{2}$. Hence, $a_{n+1} \leq b_{n+1}$.
        
        Using the lemma, we know that $a_{n} \leq b_{n}$. So, 
        \begin{align*}
            a_{n} &\leq b_{n}\\
            a_{n}a_{n} &\leq a_{n}b_{n}\\ 
            \sqrt{a_{n}a_{n}} &\leq \sqrt{a_{n}b_{n}}\\ 
            a_{n} &\leq a_{n+1}
        \end{align*}
    
        Hence, $\{a_{n}\}$ is monotonically increasing. 
    
        We want to show that $\{ b_{n} \}$ is monotonically decreasing. 
    
        We know that $a_{n} \leq b_{n}$ by the lemma. So, 
        \begin{align*}
            a_{n} &\leq b_{n} \\ 
            a_{n} + b_{n} &\leq 2b_{n} \\ 
            \frac{a_{n} + b_{n}}{2} &\leq b_{n} \\ 
            b_{n+1} &\leq b_{n}
        \end{align*}
        Therefore, $\{b_{n} \}$ is monotonically decreasing. 
    
        \item We want to show that $a_{n} \leq b_{1}$, or in other words $b_{1}$ is an upper bound for $\{a_{n}\}$.
        
        \textbf{Base Case:} $a_{1} < b_{1}$ by the assumption. 
    
        \textbf{Induction Hypothesis:} Assume that $a_{n} \leq b_{1}$ for some $n \in \mathbb{N}$.
    
        \textbf{Induction Step:} Note the following ineqaulity
        \begin{align*}
            a_{n}b_{n} \leq a_{1}b_{n} \leq a_{1}b_{1} \leq b_{1}^2
        \end{align*}
        Hence, $a_{n+1}=\sqrt{a_{n}b_{n}}\leq b_{1}$
    
        We want to show that $b_{n} \geq a_{1}$, or in other words $a_{1}$ is a lower bound for $\{b_{n}\}$.
    
        \textbf{Base Case:} $b_{1} > a_{1}$ by the assumption.
    
        \textbf{Induction Hypothesis:} Assume that $b_{n} \geq a_{1}$ for some $n \in \mathbb{N}$.
    
        \textbf{Induction Step:} Note the following inequality 
        We know that $a_{1} \leq a_{n}$ since $\{ a_{n} \}$ is monotone increasing. 
        So, $a_{1} + b_{n} \leq a_{n} + b_{n}$. So, $\frac{a_{1}+b_{n}}{2} \leq b_{n+1}$. 
        By induction hypothesis, $a_{1} \leq b_{n}$. Hence, $\frac{a_{1} + a_{1}}{2} = a_{1} \leq b_{n}+1$. 
        
        Thus $a_{1}$ is a lower bound for $\{b_{n}\}$.
    
        \item By the previous parts, we know that $\{a_{n}\}$ is bounded above and monotone increasing, so it must converge. 
        Similarly, $\{b_{n}\}$ is bounded below and monotone decreasing, so it must converge. 

        Let $a = \lim_{n \to \infty} a_{n}$ and $b = \lim_{n \to \infty} b_{n}$. 
        Then, $a = \sqrt{ab}$ and $b = \frac{a+b}{2}$. Solving for $a$ and $b$, we get $a = b$.
    \end{enumerate}

    

\end{enumerate}

\end{document}
