\documentclass[12pt]{article}

\usepackage{graphicx}			% Use this package to include images
\usepackage{amsmath}			% A library of many standard math expressions
\usepackage{amssymb}
\usepackage[margin=1in]{geometry}% Sets 1in margins. 
\usepackage{fancyhdr}			% Creates headers and footers
\usepackage{enumerate}          %These two package give custom labels to a list
\usepackage[shortlabels]{enumitem}


% Creates the header and footer. You can adjust the look and feel of these here.
\pagestyle{fancy}
\fancyhead[l]{Anthony Zhao}
\fancyhead[c]{Math 131AH Homework \#7}
\fancyhead[r]{\today}
\fancyfoot[c]{\thepage}
\renewcommand{\headrulewidth}{0.2pt} %Creates a horizontal line underneath the header
\setlength{\headheight}{15pt} %Sets enough space for the header



\begin{document} %The writing for your homework should all come after this. 
%FromSection2.4: 4*,6,15*,19,23,36*. FromSection1.1: 3,5,7,12*,17,34*.

%Enumerate starts a list of problems so you can put each homework problem after each item. 
\begin{enumerate}[start=1,label={\bfseries Problem \arabic*:},leftmargin=1in] %You can change "Problem" to be whatever label you like. 
    \item Let $\left| A \right| = n$ and $\left| B \right| = m$. 
    We want to show there existss $m^{n}$ many functions from $A$ to $B$. 

    For every element in $A$, we can map it to any of the $m$ elements of $B$. 
    Since there are $n$ elements in $A$, there are $m^{n}$ many mappings/functions from $A$ to $B$. 

    \item \begin{enumerate}
        \item We want to show that $\sim$ is an equivalence relation on $\mathbb{R}$. 
        
        \textbf{Reflexive:} For all $x \in \mathbb{R}$, $x \sim x$ since $x - x = 0$ is an integer.

        \textbf{Symmetric:} Let $x, y \in \mathbb{R}$ and $x \sim y$. Then, $x - y = a$ for some $a \in \mathbb{Z}$. 
        We also know that $-a \in \mathbb{Z}$. So, $y - x = -a$ is an integer. Thus, $y \sim x$.

        \textbf{Transitive:} Let $x, y, z \in \mathbb{R}$ and $x \sim y$ and $y \sim z$. 
        Then, $x - y = a$ and $y - z = b$ for some $a, b \in \mathbb{Z}$. Note that $a + b \in \mathbb{Z}$. 
        So, $x - z = (x - y) + (y - z) = a + b$ is an integer. Thus, $x \sim z$.

        \item Let's show that $\Phi$ is injective. Choose a $x_{1}, x_{2} \in [0, 1)$ such that $\Phi(x_{1}) = \Phi(x_{2})$. 
        Then $x_{1} - x_{2} = a$ for some $a \in \mathbb{Z}$. We also know that $\left| x_{1} - x_{2} \right| < 1$ because $x_{1}, x_{2} \in [0, 1)$. 

        Since $a$ is an integer, $a = 0$. So, $x_{1} = x_{2}$. Thus, $\Phi$ is injective. 

        Now let's show that $\Phi$ is surjective. Let $y \in \mathbb{R}$. We want to show that there exists an $x \in [0, 1)$ such that $\Phi(x) = [y]$.
        We can take the floor of $y$, to get $\lfloor y \rfloor \in \mathbb{Z}$. Subtracting $\lfloor y \rfloor$ from $y$ gives us a number in $[0, 1)$.
        So, $\Phi(y - \lfloor y \rfloor) = y$. Thus, $\Phi$ is surjective.
    \end{enumerate}
    
    \item We want to show that $f$ is bijective where 
    \[
        f(n, m) = \frac{(n+m-2)(n+m-1)}{2} +n 
    \]

    \textbf{Injective:} Injective proof from math stack exchange "Showing a function $f : \mathbb{N} \times \mathbb{N} \to \mathbb{N}$
    is injective" 

    Let $n, m, u, v \in \mathbb{N}$ such that $f(n, m) = f(u, v)$. 
    Consider two cases:

    \textbf{Case 1:} $n + m = u + v$.
    Then \[f(n, m) = \frac{(n+m-2)(n+m-1)}{2} +n = \frac{(u+v-2)(u+v-1)}{2} +u = f(u, v)\]
    
    The two products are the same by assumption, so $n = u$. This implies that $m = v$ since $n + m = u + v$. Hence, $f$ is injective. 

    \textbf{Case 2:} $n + m \neq u + v$. Without loss of generality, assume that $n + m < u + v$. 

    We note that 
    \[
    \frac{(n+m-2)(n+m-1)}{2} = \sum^{n+m-2}_{k = 1} k 
    \]

    Since this sum is just a function of $n + m$, 
    \[ 
        \frac{(n+m-2)(n+m-1)}{2} - \frac{(u+v-2)(u+v-1)}{2} = \sum^{n+m-2}_{j = 1} j - \sum^{u+v-2}_{k = 1} k  
    \]
    Expanding the $k$ sum, 

    \[
        \sum^{n+m-2}_{j = 1} j - \sum^{u+v-2}_{k = 1} k  = \sum^{n+m-2}_{j = 1} j - (\sum^{n+m-2}_{j = 1} j + \sum^{u+v-2}_{j = n+m-1} j) = -\sum^{u+v-2}_{j = n+m-1} j
    \]
    Since $n +m-2 < u + v - 2$, we can assume that the difference is at least $n + m - 1 > 0$. 
    So, \begin{align*}
        f(n, m) =  \frac{(n+m-2)(n+m-1)}{2} + n &\leq \frac{(u+v-2)(u+v-1)}{2} - (n + m - 1) + n \\ 
        &< f(u, v)
    \end{align*} 

    Thus, by trichotomy $f(n, m) \neq f(u, v)$. This contradicts the assumtion 

    \textbf{Surjective:} Notice, with some algebraic manipulation, $f(n, m) + 1 = f(1, n+1)$ if $m = 1$ and $f(n, m) + 1 = f(n+1, m-1)$ otherwise. 
    Let $k = f(n, m)$. We can induce on $k$ to show that for all $k \in \mathbb{N}$, $k = f(n, m)$ for some $n, m \in \mathbb{N}$. 

    \textbf{Base Case:} Let $k = 1$. Then, $f(1, 1) = 1$.

    \textbf{Induction Hypothesis:} Assume for some $k \in \mathbb{N}$ that $f(n, m) = k$ for some $n, m \in \mathbb{N}$.

    \textbf{Inductive Step:} Using our observation, we notice that if $k + 1 = f(n, m) + 1$ so $k + 1 = f(1, n+1)$ if $m = 1$ and $k + 1 = f(n+1, m-1)$ otherwise.
    Thus, we have found a $n, m$ such that $f(n, m) = k + 1$, so $f$ is surjective. 


    \item Let $A$ be a non-empty finite set and let $B$ be a proper subset of $A$. 
    We want to show that there cannot exist a bijection between $A$ and $B$ so they cannot have the same cardinality. 

    Assume there exists a bijection $f: A \rightarrow B$. We know that there exists a $a_{0} \in A \setminus B$. 
    Construct a sequence $\{a_{n} \}_{n\geq 0}$ where $a_{n+1} = f(a_{n})$. Note that each $a_{n}$ except $a_{0}$ is in $B$.   

    Assume that there exist some $n$ and $k$ such that $a_{n+k} = a_{n}$. This means that $f^{n}(a_{k}) = f^{n+k}(a_{0}) = f^{n}(a_{0})$. Note that the composition of bijective functions are still bijective. 
    So, by injectivity $a_{k} = a_{0}$. However, $a_{0} \not \in B$, but $a_{k} \in B$. This is a contradiction, so there cannot be repeated elements in the sequence. However, this implies that $B$ is infinite. This contradicts that $B$ is a proper subset of $A$ and is finite. Hence, there cannot exist a bijection between $A$ and $B$. Therefore, $A$ and $B$ have different cardinalities. 

    \item Assume that $A$ is a infinite set. We want to show that $A$ contains a countable subset. 
    Using an inductive argument we can select a countable subset of $A$.

    We know that $A$ is non-empty. So, we can select an element $a_{1} \in A$.
    Since $A$ is infinite, $A \setminus a_{1} \neq \emptyset$. 

    Similarly, we can select $a_{2} \in A \setminus \{ a_{1} \}$. $A \setminus \{ a_{1}, a_{2} \} \neq \emptyset$ since $A$ is infinite.

    We can continue this process, for $n$, and notice that $A \setminus \{ a_{1}, \dots, a_{n}\} \neq \emptyset$. Hence, we can select a $a_{n+1} \in A \setminus \{ a_{1}, \dots, a_{n} \}$.
    
\end{enumerate}

\end{document}
