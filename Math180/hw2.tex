\documentclass[12pt]{article}

\usepackage{graphicx}			% Use this package to include images
\usepackage{amsmath}			% A library of many standard math expressions
\usepackage{amssymb}
\usepackage[margin=1in]{geometry}% Sets 1in margins. 
\usepackage{fancyhdr}			% Creates headers and footers
\usepackage{enumerate}          %These two package give custom labels to a list
\usepackage[shortlabels]{enumitem}


% Creates the header and footer. You can adjust the look and feel of these here.
\pagestyle{fancy}
\fancyhead[l]{Anthony Zhao}
\fancyhead[c]{Math 180 Homework \#2}
\fancyhead[r]{\today}
\fancyfoot[c]{\thepage}
\renewcommand{\headrulewidth}{0.2pt} %Creates a horizontal line underneath the header
\setlength{\headheight}{15pt} %Sets enough space for the header



\begin{document} %The writing for your homework should all come after this. 
%FromSection2.4: 4*,6,15*,19,23,36*. FromSection1.1: 3,5,7,12*,17,34*.

%Enumerate starts a list of problems so you can put each homework problem after each item. 
\begin{enumerate}[start=1,label={\bfseries Problem \arabic*:},leftmargin=1in] %You can change "Problem" to be whatever label you like. 
    \item The bridges can be represented graphically like this:
    
    \includegraphics[scale=0.5]{/Users/anthonyzhao/Desktop/UCLA-Math/Math180/Images/Screenshot 2025-01-30 at 11.30.29 PM.png}
    
    \begin{enumerate}
        \item Notice that the graph has a score of $(3,3, 3, 5)$. Since, the degrees of a vertex is odd, that means there cannot exist an Eulerian Tour.
        \item Adding an additional edge between the top vertex to the bottom vertex, and one from the left vertex to the right vertex results in a graph where the degrees of all edges are even. Thus, an Eulerian Tour exists.
    \end{enumerate}

    \item  \begin{enumerate}
        \item All the graphs have Hamiltonian cycles:

        \includegraphics[scale=0.5]{/Users/anthonyzhao/Desktop/UCLA-Math/Math180/Images/Screenshot 2025-01-30 at 11.39.45 PM.png}
        \item A graph with two components of 3-cycles and a graph with one component with a 6 cycle both have a score of $(2, 2, 2, 2, 2, 2)$ but only one of them has a Hamiltonian cycle. 
    \end{enumerate}

    \item $(\Rightarrow)$ Assume $G = (V, E)$ is a directed Eulerian graph. We want to show that the symmetrization is connected and $deg_{G}^{+}(v) = deg_{G}^{-1}(v)$ for all $v \in V$. 
    
    Since there exists a directed Eulerian Tour in $G$, every vertex can be reached from every other vertex. Obviously, symmetrizing $G$ wouldn't change this fact. So, the symmetrization of $G$ is connected. 
    Assume that for some $v \in V$ that $deg_{G}^{+}(v) \neq deg_{G}^{-1}(v)$. 

    Assume that $deg_{G}^{+}(v) > deg_{G}^{-1}(v)$. Then, at some point we will enter $v$, and there will not be any out unused out vertices remaining, but there will still be vertices into $v$ remaining. This contradicts that fact that $G$ has a Eulerian Tour, since all vertices have not been used. 

    The case for $deg_{G}^{+}(v) < deg_{G}^{-1}(v)$ follows similarly. There will be a point where we leave $v$ with other out edges remaining, but not be able to return to $v$. Hence, by contradiction, $deg_{G}^{+}(v) = deg_{G}^{-1}(v)$. 

    $(\Leftarrow)$ Assume that the symmetrization of $G$ is connected and $deg_{G}^{+}(v) = deg_{G}^{-1}(v)$ for all $v \in V$. We want to show that $G$ is a directed Eulerian graph.

    Let $T$ be a tour on $G$ of maximal length $(n)$. So, 
    \[
        T = (v_{0}, e_{1}, \dots, e_{n}, v_{n})
    \]
    We will show that this tour is a Eulerian Tour. 

    First, we want to show that $v_{0} = v_{n}$. 

    Assume that $v_{0} \neq v_{n}$. This means we've exited $v_{0}$ more times than we have entered $v_{m}$. By assumption, $deg_{G}^{+}(v) = deg_{G}^{-1}(v)$, so 
    there exists an edge into $v_{0}$, such that we can extend the walk. Hence, $T$ is not maximal. By contradiction, $v_{0}= v_{n}$ in a maximal tour.

    Next, we want to show that all vertices are in $T$. 
    By assumption that $G$ is weakly connected, every vertex must have at least one edge going out of it or going in it. 
    Assume that $v \not \in T$.
    Since $deg_{G}^{+}(v) = deg_{G}^{-1}(v)$, this guarantees that there must be at least a pair of edges going in and out of $v$. 
    Hence, we can extend the tour by adding $v$ to the tour. Thus, by contradiction, all vertices are in $T$ if it is maximal.

    Lastly, assume that there exists an edge $e \in E$ that is not in $T$. 
    Then we can add $e$ to $T$ to extend the tour, and use the previous two claims to show that $T$ is not the maximal tour. 

    Hence, $T$ is an Eulerian Tour since $T$ visits every edge, vertex, and starts and ends in the same place.

    \item I am pretty sure that the forward direction is not correct. If we assume that $G$ is a strongly directed graph and has a cycle of even length, it doesn't necessarily have to be 2-colorable. 
    For example, consider the following graph:

    \includegraphics[scale=0.5]{/Users/anthonyzhao/Desktop/UCLA-Math/Math180/Images/Screenshot 2025-01-31 at 4.42.58 PM.png}
    
    This graph has a cycle of length 4 and is strongly connected, but it is not 2-colorable.

    If we change the first claim to be that every cycle in $G$ is an even cycle. The equivalence holds true. 

    $(\Rightarrow)$ Assume that every cycle in $G$ is an even cycle. We want to show that $G$ is two colorable. 
    
    Take a vertex $v$. We know that for every $v'$, there exists a path from $v$ to $v'$ (and $v'$ to $v$, which creates a cycle). Take the shortest path from $v$ to $v'$. 
    If the length of this path is even, then color it the same color as $v$. If it is odd, then color it the other color. 

    Why does this work? 

    Let $P$ be the path from $v$ to $v'$, and let $P'$ be the path from $v$ to $v'$. Assume that $P$ is an odd length. We can define a cycle from $v$ to $v'$ using $P$ and then $v'$ to $v$ using $P'$. 
    If $P$ is of odd length, then $P'$ must also be of odd length, since cycles are of even length, which stays consistent with the coloring. It is similar if $P$ is of even length as well. The coloring stays consistent in either case. 

    


    $(\Leftarrow)$ Assume that $G$ is two colorable. We want to show that every cycle is an even cycle. 
    Assume otherwise. There exists a odd cycle $C = (v, v_{1}, \dots, v_{n}, v_{1})$. Notice that every vertex an odd distance away from $v_{1}$ is the same color as $v_{1}$. 
    By assumption $v_{n}$ is an odd distance from $v_{1}$. So it is the same color as $v_{1}$. However, since $v_{n}$ and $v_{1}$ are connected, they must be different colors by assumption. 
    Hence, by contradiction, there cannot exist a cycle $C$ of odd cycle. So, every cycle in $G$ is an even cycle. 


    \item Let $G$ be a tournament. We want to show that $G$ has a directed path passing through all vertices (Hamiltonian Path). 
    
    We will prove this by induction on the number of vertices in $G$.

   \textbf{Base Case:} $|V| = 1$. Then, the graph is trivially a Hamiltonian Path. $V = 2$ is also a Hamiltonian path since there is one directed edge between two vertices, so there must be a path that visits both vertices. 

    \textbf{Inductive Hypothesis:} Let $G = (V, E)$ be a tournament where $|V| = n$ for some $n \in \mathbb{N}$. Assume for all such $G$, there exists a directed path $P$ that contains all the vertices of $G$. 

    \textbf{Induction Step:} Let $G$ be a tournament with $|V| = n+1$. We want to show there exists a directed path $P$ that contains all the vertices of $G$. 

    \textbf{Lemma:} $G$ is a tournament $\Longleftrightarrow$ $G-v$ is a tournament for any $v$ in $V$. 

    Remove any $v$ from $G$ to ge ta tournament $G'$. Note that $G'$ has $n$ vertices, so we can use the induction hypothesis and find some directed path $P'$ that contains all the vertices of $G'$. 
    We want to show that we can add $v$ somewhere into the path and still form a directed path with all the vertices. 
    \[
        P' = (v_{1}, v_{2}, \dots, v_{n})
    \]

    \textbf{Case 1:} For all $v' \in V'$, there exists an edge $(v, v') \in E$. In other words, all the edges are coming out of $v$ to $v'$. 
    Then, we can append $v$ to the beginning of the path $P'$ to form $P = (v, v_{1}, \dots, v_{n})$, and this is a path containing all vertices in $G$. 

    \textbf{Case 2:} There exists a $v_{k} \in V'$ such that $(v_{k}, v) \in E$. Select the last such $v_{k}$ in $P$. Then we know that $(v, v_{k+1}) \in E$ (if $k \neq n$). 
    So, we can append $v$ after $v_{k}$ to form $P = (v_{1}, \dots, v_{k}, v, v_{k+1}, \dots, v_{n})$. This is a path containing all vertices in $G$.
    If $k = n$, then we can append $v$ to the end of the path. In both cases, $P$ is a directed path that contains all the vertices in $G$. 
    
\end{enumerate}

\end{document}
