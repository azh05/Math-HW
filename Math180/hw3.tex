\documentclass[12pt]{article}

\usepackage{graphicx}			% Use this package to include images
\usepackage{amsmath}			% A library of many standard math expressions
\usepackage{amssymb}
\usepackage[margin=1in]{geometry}% Sets 1in margins. 
\usepackage{fancyhdr}			% Creates headers and footers
\usepackage{enumerate}          %These two package give custom labels to a list
\usepackage[shortlabels]{enumitem}


% Creates the header and footer. You can adjust the look and feel of these here.
\pagestyle{fancy}
\fancyhead[l]{Anthony Zhao}
\fancyhead[c]{Math 180 Homework \#1}
\fancyhead[r]{\today}
\fancyfoot[c]{\thepage}
\renewcommand{\headrulewidth}{0.2pt} %Creates a horizontal line underneath the header
\setlength{\headheight}{15pt} %Sets enough space for the header



\begin{document} %The writing for your homework should all come after this. 
%FromSection2.4: 4*,6,15*,19,23,36*. FromSection1.1: 3,5,7,12*,17,34*.

%Enumerate starts a list of problems so you can put each homework problem after each item. 
\begin{enumerate}[start=1,label={\bfseries Problem \arabic*:},leftmargin=1in] %You can change "Problem" to be whatever label you like. 
    \item \textbf{Lemma:} Removing a leaf from a tree (with at least 2 vertices) creates a new tree. I'm pretty sure this was proven in class, but the idea is that the tree still remains connected because it was a leaf, and there is still no cycles because there were non to begin with. 
    
    We will induce on the number of vertices in the rree. 

    \textbf{Base Case:} Let $T$ be a tree with two vertices. The only such tree has two vertices, both of degree 1. So the sum would equal to $p_{1} = 2$. 

    \textbf{Induction Hypothesis:} Assume that the equality holds for any tree with $n$ vertices.

    \textbf{Induction Step:} Let $T$ be a tree with $n+1$ vertices. Let $v$ be a leaf of $T$. Then, $T - v$ is a tree with $n$ vertices. 
    Hence, the equality 
    \[
        p_{1}' - p_{3}' - 2 p_{4}' - \dots - (n-3)p_{n-1}' = 2
    \]
    holds. Let's add back the leaf $v$ to $T - v$. 

    \textbf{Case 1:} The leaf was added to another leaf. Then, $p_{n} = 0$, and $p_{1} = p_{1}'$. This is because the degree of the old leaf is now 2, and the degree of the new leaf is 1, so the number of vertices of degree 1 remains the same.
    The number of vertices of the other degrees remain the same, so $p_{i} = p_{i}'$. 
    So, 
    \[
        p_{1} - p_{3} - 2 p_{4} - \dots - (n-3)p_{n-1} - (n-2)p_{n} =  p_{1}' - p_{3}' - 2 p_{4}' - \dots - (n-3)p_{n-1}' = 2
    \]

    \textbf{Case 2:} The leaf was added back to the tree to a vertex $u$ of degree $k > 1$. Then, $p_{1} = p_{1}' + 1$ because we add a leaf, and the degree of the vertex that the leaf was added to is not of degree 1. 
    Now, realize that the $u$ was adding $-(k)$ to the sum in $T'$. But, now the vertex is adding $-(k+1)$ because an extra vertex was added. The difference in the previous and new sum of $p_{k} + p_{k+1}$ is $-1$.
    Combining these two results, we find that $-1 + 1 = 0$, so there is no change in the sum 
    \[
        p_{1} - p_{3} - 2 p_{4} - \dots - (n-3)p_{n-1} - (n-2)p_{n} -  p_{1}' - p_{3}' - 2 p_{4}' - \dots - (n-3)p_{n-1}' = 0
    \]
    Hence, 
    \[
        p_{1} - p_{3} - 2 p_{4} - \dots - (n-3)p_{n-1} - (n-2)p_{n} = 2 
    \]
    \item These two trees both have score of $(1, 1, 2, 2, 3)$ but are not isomorphic 
    
    \includegraphics[scale=0.5]{/Users/anthonyzhao/Desktop/UCLA-Math/Math180/Images/Screenshot 2025-02-07 at 12.09.10 AM.png}

    \item By what we have proven in class, there is a unique encoding of every tree, that uses two bits to encode information about every vertex. 
    Hence, if there are $n$ vertices, there is a unique $2n$ string of bits to encode that tree. There are $2^{2n}$ possible strings of bits, so there are $2^{2n} = 4^{n}$ possible trees.
\end{enumerate}

\end{document}
