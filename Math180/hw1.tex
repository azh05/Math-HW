\documentclass[12pt]{article}

\usepackage{graphicx}			% Use this package to include images
\usepackage{amsmath}			% A library of many standard math expressions
\usepackage{amssymb}
\usepackage[margin=1in]{geometry}% Sets 1in margins. 
\usepackage{fancyhdr}			% Creates headers and footers
\usepackage{enumerate}          %These two package give custom labels to a list
\usepackage[shortlabels]{enumitem}


% Creates the header and footer. You can adjust the look and feel of these here.
\pagestyle{fancy}
\fancyhead[l]{Anthony Zhao}
\fancyhead[c]{Math 180 Homework \#1}
\fancyhead[r]{\today}
\fancyfoot[c]{\thepage}
\renewcommand{\headrulewidth}{0.2pt} %Creates a horizontal line underneath the header
\setlength{\headheight}{15pt} %Sets enough space for the header



\begin{document} %The writing for your homework should all come after this. 
%FromSection2.4: 4*,6,15*,19,23,36*. FromSection1.1: 3,5,7,12*,17,34*.

%Enumerate starts a list of problems so you can put each homework problem after each item. 
\begin{enumerate}[start=1,label={\bfseries Problem \arabic*:},leftmargin=1in] %You can change "Problem" to be whatever label you like. 
    \item (4.1.1)
    \begin{enumerate}
        \item Label the two graphs with the following:
    
        \includegraphics[scale=0.8]{~/Desktop/UCLA-Math/Math180/Images/Screenshot 2025-01-24 at 12.13.23 PM.png}
        
        There exists a bijection between the graphs 
        \[
        \begin{pmatrix}
            A & B & C & D & E  & F & G & H & I & J  \\
            1 & 2 & 3 & 4 & 10 & 9 & 6 & 8 & 5 & 7
        \end{pmatrix}
    \]
    \item At a glance, each vertex is connected to three other vertices, since $v = \{ a, b\}$ is connected to the two element subsets of $\{ 1, 2, 3, 4, 5 \} \setminus \{a, b\}$ and there are 10 vertices, both properties match the above two graphs.
    More explicitly, we can construct the following  bijection

    \[
        \begin{pmatrix}
            A & B & C & D & E  & F & G & H & I & J  \\
            \{ 1, 2 \} & \{ 4, 5 \} & \{ 2, 3 \} & \{ 1, 5 \} & \{ 3, 4 \} & \{ 3, 5 \} & \{ 1, 3 \} & \{ 1, 5 \} & \{ 2, 4 \} & \{ 2, 5 \} 
        \end{pmatrix}
    \]
    \end{enumerate}
    
    \item (4.1.3)
    \begin{enumerate}
        \item \includegraphics[scale=0.5]{/Users/anthonyzhao/Desktop/UCLA-Math/Math180/Images/Screenshot 2025-01-24 at 12.27.16 PM.png}
        \item \textbf{Lemma 1:} Graphs with maximum degree 1 have a nontrivial automorphism. The graph would consist of connected components of two connected vertices, $v_{1}, v_{2} \in V$. Mapping $v_{1}$ to $v_{2}$ and vice versa would be an isomorphism
    
        \textbf{Lemma 2:} Graphs with maximum degree 2 have a non-trivial automorphism. 
        These types of graphs can be split into the case where the graph has a cycle where vertices in the cycle are degree 2 and the case where the connected components are paths, with two vertices of degree 1. 
       
        (Cycle Case) If we start from any given vertex, $v_{0}$, it must be connected to two other vertices, $v_{1}, v_{2}$. We can map $v_{1}$ to $v_{2}$ and $v_{2}$ to $v_{1}$. These vertices must also be connected to another vertex $v_{3}$ and $v_{4}$ respectively (where $v_{3}$ isn't necessarily different from $v_{4}$). 
        We can map $v_{3}$ to $v_{4}$ and $v_{4}$ to $v_{3}$. If we iteratively do this, we always get an isomorphism. 
    
        (Path Case) Let $P$ be the path. Start from the two vertices, $v_{1}, v_{n}$ with degree $1$. We can map $v_{1}$ to $v_{n}$ and vice versa. Take those two elements out of the path, and then we get two more vertices of degree 1. Iteratively, do this and we get another non-trivial automorphism. 

        Using Lemmas $1$ and $2$, we know that any graph with $3$ or less vertices have nontrivial automorphisms.
        
        \textbf{Lemma 3:} Disconnected graphs have non-trivial automorphisms. Intiutively, we can map two separate connected components to each other, and we still get an isomorphism. 
         
        An exhaustive search of all connected graphs with 4 and 5 vertices, whose max degree is at least 3 reveals that there doesn't exist a asymmetric graph. 

    \end{enumerate}
    
    \item (4.2.1) Let $G = (V, E)$ be a graph with $k$ connected components ($k > 1$). 
    Let $G' = (V, E' = \binom{V}{2} \ E)$ denote the complement of $G$. 

    Let $C_{1}$ and $C_{2}$ be distinct connected components of $G$. Then, by definition that for all $v_{1} \in C_{1}$ and $v_{2} \in C_{2}$ there does not exist $\{ v_{1}, v_{2} \} \in E$. 
    Hence, for all $v_{1} \in C_{1}$ and $v_{2} \in C_{2}$, there exists $\{ v_{1}, v_{2} \} \in E'$. This shows that every connected component in $G$ is connected to each other in $G'$. 

    Now, we want to show that there still remains a path within every connected component in $G'$. 
    Let $C_{1} = (V_{1}, E_{1})$ and $C_{2} = (V_{2}, E_{2})$ be connected components of $G$. Let $w_{1}, w_{2} \in V_{1}$ be distinct vertices in $w_{1}$ and $w_{2}$. 
    We know that there is an edge between every vertex in $C_{1}$ and every vertex in $C_{2}$ in $G'$. Let $x \in V_{2}$. So, $\{ w_{1}, x \} \in E'$ and $\{ w_{2}, x \} \in E'$. 
    Thus, we can form the path $(w_1 \: x \: w_{2})$. Therefore, every connected component in $G$ remains connected in $G'$. 

    Therefore, since each connected component remains connected and there exists a path between connected components, then any two vertices in $G'$ can be connected by a path. Therefore, $G'$ is a connected graph. 

    \item (4.2.5) Let $G$ be a graph containing no path of length 3 (this implies that $G$ doesn't have a path of longer than 3). 
    
    Then, each connected component of $G$ must look like the following. 

    \includegraphics[scale = 0.5]{/Users/anthonyzhao/Desktop/UCLA-Math/Math180/Images/Screenshot 2025-01-24 at 12.59.51 PM.png}

    \item (4.3.1) 
    The first graph has 5 cycles of length 4. The second graph has 2 cycles of length 4. The third graph has 0 cycles of length 4. Hence, the graphs cannot be isomorphic. 

    \item (4.3.9) Let $G = (V, E)$ be a graph where all vertices have at least degree $d$. We want to  show that $G$ contains a path of length $d$. 
    
    \textbf{Lemma:} The complete graph $K_{n+1}$ is a subgraph of any graph with degree at least $n$. 

    \textbf{Idea:} Each vertex in $K_{n+1}$ has degree $n$. We can continuously remove edges and vertices from the graph until we obtain $K_{n+1}$. 

    Hence, if we show that $K_{n+1}$ has a path of length $n$, then every graph that has degree at least $n$ will have a path of length $n$. 
    
    \textbf{Proof By Induction:} 

    \textbf{Base Case:}
    Let $p = 1$. Evidentally, $K_{1+1} = K_{2}$ has a path of length 1, since it is a graph with two vertices, connected by a edge. 

    \textbf{Induction Hypothesis:}
    Let $p = n$ for some $n \in \mathbb{N}$. Assume that $K_{n+1}$ has a path of length $n$. Let this path be $P = (v_{0} \: e_{1} \: \dots e_{n} v_{n})$

    \textbf{Induction Step:} We want to show that $K_{n+2}$ has a path of length $n+1$. 
    We can construct $K_{n+2}$ from $K_{n+1}$ by adding a vertex $v_{n+1}$, and connecting the vertex to every $v \in K_{n+1}$. Since $v_{n+1}$ is connected to $v_{n}$ and $v_{n+1}$ is not in $K_{n+1}$,
    we can add $v_{n+1}$ to the path $P$, and it remains a path, $(v_{0} \: e_{1} \: \dots e_{n} v_{n} \: e_{n+1} v_{n+1})$. Note that this path is now of length $n+1$. 
    

\end{enumerate}

\end{document}
