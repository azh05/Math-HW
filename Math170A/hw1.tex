\documentclass[12pt]{article}

\usepackage{graphicx}			% Use this package to include images
\usepackage{amsmath}			% A library of many standard math expressions
\usepackage{amssymb}
\usepackage[margin=1in]{geometry}% Sets 1in margins. 
\usepackage{fancyhdr}			% Creates headers and footers
\usepackage{enumerate}          %These two package give custom labels to a list
\usepackage[shortlabels]{enumitem}


% Creates the header and footer. You can adjust the look and feel of these here.
\pagestyle{fancy}
\fancyhead[l]{Anthony Zhao}
\fancyhead[c]{Math 170A Homework \#1}
\fancyhead[r]{\today}
\fancyfoot[c]{\thepage}
\renewcommand{\headrulewidth}{0.2pt} %Creates a horizontal line underneath the header
\setlength{\headheight}{15pt} %Sets enough space for the header



\begin{document} %The writing for your homework should all come after this. 
%FromSection2.4: 4*,6,15*,19,23,36*. FromSection1.1: 3,5,7,12*,17,34*.

%Enumerate starts a list of problems so you can put each homework problem after each item. 
\begin{enumerate}[start=1,label={\bfseries Problem \arabic*:},leftmargin=1in] %You can change "Problem" to be whatever label you like. 
    \item Let $\mathcal{B} = \mathcal{P}(\{P_1, ..., P_n\})$ be the power set of the partitions of $\Omega$. 
    We can construct $\mathcal{F}_P$ to be the union of the partitions in every set in $\mathcal{B}$. 
    Or \[
        \mathcal{F}_P = \{ \bigcup_{i=1}^j a_i | a_i \in A, A \in \mathcal{B}, \textnormal{ where } |A| = j\}
    \]
    This is the smallest $\sigma-$algebra containing each of the subsets $P_i$. 

    \begin{enumerate}
        \item $\mathcal{F}_P$ is non-empty because $\mathcal{B}$ contains $\emptyset$, so $\mathcal{F}_P$ does as well.
        \item $\mathcal{F}_P$ is compliment closed. Let $B \in \mathcal{F}_P$.
        Then, $B = \bigcup_{i\in \mathcal{J}} P_i$ for some $\mathcal{J} \subseteq \{1, ..., n\}$ by how $\mathcal{F}_P$ is constructed. 
        
        We know that $B \cup B^C = \Omega$. So 
        \[
            (\bigcup_{i\in \mathcal{J}} P_i) \cup B^C = \Omega
        \]

        Since the the $P_i$'s are partitions, then 
        \[
            B^C = \bigcup_{j \not \in \mathcal{J}} P_j
        \]

        We know that $ \bigcup_{j \not \in \mathcal{J}} P_j$ is in $\mathcal{F}_{P}$ 
        by construction. So, $B^C \in \mathcal{F}_P$ and $\mathcal{F}_P$ is compliment closed. 

        \item By construction, $\mathcal{F}_P$ is countable $\cup$ closed
        \item Since $P_{i} \cap P_{j} = \emptyset$ for $i \neq j$, $\mathcal{F}_{P}$ is countable $\cap$ closed 
    \end{enumerate}    
    Since there are $2^n$ elements in the power set $\mathcal{B}$ and there is a one to one correspondence
    between elements in $\mathcal{B}$ and $\mathcal{F}_{P}$, there are $2^n$ subsets of $\Omega$ in $\mathcal{F}_P$. 
    
    Also, this must be the smallest $\sigma$-algebra because the smallest sigma algebra only contains the union of all the partitions of $\Omega$. 
    \item $(\Rightarrow)$ Assume that $f$ is measurable. We know for all $A \in \mathcal{E}$, $f^{-1}(A) \in \mathcal{F}_{P}$.

    Let $e \in E$. Since $\mathcal{E}$ is a power set, $\{e\} \in \mathcal{E}$. 
    So $f^{-1}(\{e\}) \in \mathcal{F}_{P}$ and $f^{-1}(\{e\}) = \bigcup_{i\in \mathcal{I}} P_i$ for some $\mathcal{I} \subseteq \{ 1, ..., n\}$. 
    Let $P_j \subseteq \bigcup_{i\in \mathcal{I}} P_i$ for the same index set $\mathcal{I}$. Then, for all $p \in P_j$, $f(p) = e$. 
    Therefore the restriction of $f$ to each $P_j$ is a constant function. 


    $(\Leftarrow)$ Assume that for each $i = 1, ..., n$, $f|_{P_i}$ is a constant function. We want to show that $f$ is measurable. 
    
    Let $A \in \mathcal{E}$. Let $A' \subseteq A$ where $A' = A \cap Im(f)$. Then, $f^{-1}(A') = \bigcup  \{ P_i | f(P_i) = e, \forall e \in A' \}$ or in other words 
    union of the set of partitions whose image maps to an element of $A$. 
    Note that $f^{-1}(A') = f^{-1}(A)$ because the elements not in $Im(f)$ map to the empty set, and the union of the empty set to another set $S$ is $S$. 

    Since $\mathcal{F}_P$ contains all the unions of the partitions, $f^{-1}(A) \in \mathcal{F}_P$.


    \item $(\Rightarrow)$ Assume $X_{1}$ is measurable with respect to $\mathcal{F}_{X_{0}}$. 
    So, we know that for all $A \in \mathcal{E}_{1}$, $X_{1}^{-1}(A) \in \mathcal{F}_{X_{0}}$. 
    Since $\mathcal{F}_{X_{0}} := \{X_{0}^{-1}(B) | B \in \mathcal{E}_{0}\ \}$ that means 
    for all $A$ there exists a $B \in \mathcal{E}_{0}$ such that $X_{1}^{-1}(A) = X_{0}^{-1}(B)$. 
    
    Therefore, we can define $h$ by the following. Let $x_{0} \in X_{0}$. 
    Then, $h(x_{0}) = X_{1}(\omega)$ for some $\omega \in \Omega$ where $X_{0}(\omega) = x_{0}$. We can assume that this $\omega$ exists by the problem statement. 

    $(\Leftarrow)$ Let $A \in \mathcal{E}_{1}$. We know that $X_{1}^{-1}(A) = (h\circ X_{0})^{-1}(A)$. 
    
    The preimage of $h\circ X_{0}$ is defined by the following 
    \[
        (h\circ X_{0})^{-1}(A) = \{ \omega \in \Omega | h(X_{0}(\omega)) \in A \}
    \]

    Another way to write this is 
    \[
        = \{ \omega \in \Omega | X_{0}(\omega) \in h^{-1}(A) \}
    \]
    where 
    \[
        h^{-1}(A) = \{ x_{0} \in E_{0} | h(x_{0}) \in A\}
    \]
    This shows that $h^{-1}(A) \subseteq E_{0}$. Since, $(E_{0}, \mathcal{E}_{0})$ is a discrete space, 
    every subset of $E_{0}$ must be contained in the space. Hence, $h^{-1}(A) \in \mathcal{E}_{0}$
    and $X_{0}^{-1}(h^{-1}(A)) \in \mathcal{F}_{X_{0}}$

    Note, $X_{0}^{-1} \circ h^{-1} = X_{1}^{-1}$. Hence, $X_{1}^{-1}(A) \in \mathcal{F}_{X_{0}}$. Therefore, $X_{1}$ is measurable with respect to $\mathcal{F}_{X_{0}}$

    \item \begin{enumerate}
        \item $\mathbb{P}_{X}$ is non-negative.
        
            Let $A \in \mathcal{E}$. Because $X$ is a measureable function, we know that $X^{-1}(A) \in \mathcal{F}$. 
            Since $\mathbb{P}$ is a probability measure over $(\Omega, \mathcal{F})$ that means $\mathbb{P}(X^{-1}(A)) \geq 0$. Thus,
            for all $A \in \mathcal{E}$, $\mathbb{P}_{X}(A) = \mathbb{P}(X^{-1}(A)) \geq 0$.
        \item $\mathbb{P}_{X}$ satisifes countable additivity. 
        
        First, we prove that the pre-image of disjoint sets are also disjoint. 

        Let $A_{i}, A_{j} \in \mathcal{E}$ and $A_{i} \cap A_{j} = \emptyset$. Then,
        by definition of preimage we know that 
        \begin{align*}
            X^{-1}(A_i) &= \{ \omega \in \Omega | X(\omega) \in A_{i}\}\\
            X^{-1}(A_j) &= \{ \omega \in \Omega | X(\omega) \in A_{j}\}
        \end{align*}
        
        Let $\omega \in X^{-1}(A_i) \cap X^{-1}(A_j)$. Then, $X(\omega) \in A_{i} \cap A_{j}$. 
        However, we assume that $A_{i} \cap A_{j} = \emptyset$. Thus, there cannot exist such $\omega$, showing that $X^{-1}(A_i) \cap X^{-1}(A_j) \subseteq \emptyset$ and $X^{-1}(A_i) \cap X^{-1}(A_j) = \emptyset$
        
        Another useful fact to know is that 
        \[
            X^{-1}(\bigcup_{i \in \mathcal{I}}A_{i}) = \bigcup_{i \in \mathcal{I}} X^{-1}(A_{i})
        \]
        
        This can be proven just by stating the definition of the two sides. 
        Using these two facts, we get that
        \[
            X^{-1}(\bigsqcup_{i \in \mathcal{I}}A_{i}) = \bigsqcup_{i \in \mathcal{I}} X^{-1}(A_{i})
        \]
        
        Now, we want to show that for $A_1, A_{2}, \dots \in \mathcal{E}$ such that $A_{i} \cap A_{j} = \emptyset$ for $i \neq j$, that 
        \[
            \mathbb{P}_{X}(\bigsqcup_{i \in \mathcal{I}}A_{i}) = \sum_{i \in \mathcal{I}} \mathbb{P}_{X}(A_{i})
        \]
        Using the definition of $\mathbb{P}_{X}$,
        \[
            \mathbb{P}_{X}(\bigsqcup_{i \in \mathcal{I}}A_{i}) = \mathbb{P}(X^{-1}(\bigsqcup_{i \in \mathcal{I}}A_{i}))
        \]
        Using our lemma, 
        \[
            \mathbb{P}(X^{-1}(\bigsqcup_{i \in \mathcal{I}}A_{i})) = \mathbb{P}(\bigsqcup_{i \in \mathcal{I}}X^{-1}(A_{i}))
        \]
        Since, $X^{-1}(A_{i}) \in \mathcal{F}$ because $X$ is measurable, 
        \[
            \mathbb{P}(\bigsqcup_{i \in \mathcal{I}}X^{-1}(A_{i})) = \sum_{i \in \mathcal{I}} \mathbb{P}(X^{-1}(A_{i}))
        \] 
        Using the definition of $\mathbb{P}$, we find
        \[
            \sum_{i \in \mathcal{I}} \mathbb{P}(X^{-1}(A_{i})) = \sum_{i \in \mathcal{I}}\mathbb{P}_{X}(A_{i})
        \]
        Thus, $\mathbb{P}_{X}$ is countable additive.

        \item $\mathbb{P}_{X}(\emptyset) = 0$. We know that $X^{-1}(\emptyset_{\mathcal{E}}) = \emptyset_{\Omega}$.
        So, $\mathbb{P}(\emptyset_{\Omega}) = \mathbb{P}(X^{-1}(\emptyset_{\mathcal{E}})) = \mathbb{P}_{X}(\emptyset_{\mathcal{E}}) = 0$
        \item $\mathbb{P}_{X}(E) = 1$. We know that 
        \[
        X^{-1}(E) = \{ \omega \in \Omega | X(\omega) \in E\} 
        \]
        By the definition of $X$, for all $\omega$, $X(\omega) \in E$, so $X^{-1}(E) = \Omega$. 
        Hence, $\mathbb{P}_{X}(E) = \mathbb{P}(X^{-1}(E)) = \mathbb{P}(\Omega) = 1$
    \end{enumerate}

\end{enumerate}

\end{document}