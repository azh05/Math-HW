\documentclass[12pt]{article}

\usepackage{graphicx}			% Use this package to include images
\usepackage{amsmath}			% A library of many standard math expressions
\usepackage{amssymb}
\usepackage[margin=1in]{geometry}% Sets 1in margins. 
\usepackage{fancyhdr}			% Creates headers and footers
\usepackage{enumerate}          %These two package give custom labels to a list
\usepackage[shortlabels]{enumitem}


% Creates the header and footer. You can adjust the look and feel of these here.
\pagestyle{fancy}
\fancyhead[l]{Anthony Zhao}
\fancyhead[c]{Math 170A Homework \#2}
\fancyhead[r]{\today}
\fancyfoot[c]{\thepage}
\renewcommand{\headrulewidth}{0.2pt} %Creates a horizontal line underneath the header
\setlength{\headheight}{15pt} %Sets enough space for the header



\begin{document} %The writing for your homework should all come after this. 
%FromSection2.4: 4*,6,15*,19,23,36*. FromSection1.1: 3,5,7,12*,17,34*.

%Enumerate starts a list of problems so you can put each homework problem after each item. 
\begin{enumerate}[start=1,label={\bfseries Problem \arabic*:},leftmargin=1in] %You can change "Problem" to be whatever label you like. 
    \item At time step $t$, $V_{x} = \{ -t, -t + 2, \dots, t-2, t\}$ and $\mathbb{P}(x = -t + 2k) = \binom{t}{k}p^{k}q^{t-k}$ because to be at step $-t + 2k$, you have to choose $+1$ $k$ times out of $t$ 
    
    \item We are given that $Z_{t}$ are all mutually independent. 
    \[
        X_{t} - X_{s_{1}} = \sum^{t}_{{i = s_{1} + 1}} Z_{i}
    \]
    \[
        X_{s_{1}} - X_{s_{0}} = \sum^{s_{1}}_{i=s_{0}+1}Z_{i}
    \]

    So, the random variable $X_{t} - X_{s_{1}}$ dependent on $\{ Z_{s_{1+1}}, \dots, Z_t\}$ 
    while the random variable $X_{s_{1}} - X_{s_{0}}$ dependents on $\{ Z_{s_{0}+1}, \dots, Z_{s}\}$. Notice that these two sets are 
    disjoint and are constructed from mutually independent events . By some theorem or by intuition, since the events are all mutually independent and the sets are disjoint, 
    the sets of events are also independent (think about how the union of the two sets can be separated into the probabilities of each set occuring). Hence $X_{t} - X_{s_{1}}$ and $X_{s_{1}} - X_{s_{0}}$ are independent random variables

    \item From question 2, we know that $X_{t} - X_{s_{1}}$ and $X_{s_{1}} - X_{s_{0}}$ are independent random variables. So, by the definition of independence, 
    \[\mathbb{P}(X_{t}-X_{s_{1}} = x_{t} - x_{s_{1}} | X_{s_{1}} - X_{s_{0}} = x_{s_{1}} - x_{s_{0}}) = \mathbb{P}(X_{t}-X_{s_{1}} = x_{t} - x_{s_{1}})\]

    Intuitively, this equation is equivalent to 
    \[
        \mathbb{P}(X_{t}-X_{s_{1}} = x_{t} - x_{s_{1}} | X_{s_{1}} = x_{s_{1}},  X_{s_{0}} = x_{s_{0}}) = \mathbb{P}(X_{t}-X_{s_{1}} = x_{t} - x_{s_{1}})
    \]
    because the given on the left side of the equation can be derived from knowing the values of the random variables $X_{s_{1}}, X_{s_{0}}$. 
    Since, we are given that $X_{s_{1}} = x_{s_{1}}$, it is redundant on the left side of the probabilities. So, intuitively, this is equivalent to 
    \[
        \mathbb{P}(X_{t} = x_{t} | X_{s_{1}} = x_{s_{1}},  X_{s_{0}} = x_{s_{0}}) = \mathbb{P}(X_{t} = x_{t} | X_{s_{1}} = x_{s_{0}})
    \]
    

    \item \begin{itemize}
        \item Note that for $x  \not \in \{ x_{t-1} + 1, x_{t_{1}} - 1\}$, $\mathbb{P}(X_{t} = x | X_{t-1} = x_{t-1}) = 0$. 
        Therefore, the sum on the right side is equal to 
        \begin{align*}
            &\mathbb{P}(X_{t+1} = x_{t+1} | X_{t} = x_{t-1} + 1)\mathbb{P}(X_{t} = x_{t-1} + 1 | X_{t-1} = x_{t-1}) \\
            &+ \mathbb{P}(X_{t+1} = x_{t+1} | X_{t} = x_{t-1} - 1)\mathbb{P}(X_{t} = x_{t-1} - 1 | X_{t-1} = x_{t-1})
        \end{align*}
            
        Assume that $X_{t+1} = x_{t-1} + 2$. Then, $\mathbb{P}(X_{t+1} = x_{t-1} + 2| X_{t} = x_{t-1} - 1) = 0$. So, 
        $\mathbb{P}(X_{t+1} = x_{t-1} + 2 | X_{t-1} = x = t_{1}) = p \cdot p + 0 = p^{2}$.  

        For $X_{t+1} = x_{t-1} - 2$, $\mathbb{P}(X_{t+1} = x_{t-1} - 2| X_{t} = x_{t-1} + 1) = 0$. So, 
        $\mathbb{P}(X_{t+1} = x_{t-1} - 2 | X_{t-1} = x = t_{1}) = 0 + q \cdot q = q^{2}$.  


        For $X_{t+1} = x_{t-1}$, we can get to $X_{t+1} = x_{t-1}$ from $X_{t-1} = x_{t-1}$ by subtracting one first and then adding one, and then reversing the order. 
        Hence, $\mathbb{P}(X_{t+1} = x_{t-1} | X_{t-1} = x = t_{1}) = p(1-q) + (1-q)p = 2p(1-q)$. 

        Therefore, the sum is equal to the piece wise function. 

        \item The general form for $n$ time steps is 
        \[
            \mathbb{P}(X_{t+n} = x_{t+n} | X_{t} = x_{t}) = \sum_{x_{t} \in \mathbb{Z}} \sum_{x_{t+1} \in \mathbb{Z}} \cdots \sum_{x_{t+n-1} \in \mathbb{Z}} \prod_{k = t + 1}^{t+n} \mathbb{P}(X_{k} = x_{k}| X_{{k-1}} = x_{k-1} )
        \] 
    \end{itemize}
\end{enumerate}

\end{document}
