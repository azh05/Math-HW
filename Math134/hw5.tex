\documentclass[12pt]{article}

\usepackage{graphicx}			% Use this package to include images
\usepackage{amsmath}			% A library of many standard math expressions
\usepackage{amssymb}
\usepackage[margin=1in]{geometry}% Sets 1in margins. 
\usepackage{fancyhdr}			% Creates headers and footers
\usepackage{enumerate}          %These two package give custom labels to a list
\usepackage[shortlabels]{enumitem}


% Creates the header and footer. You can adjust the look and feel of these here.
\pagestyle{fancy}
\fancyhead[l]{Anthony Zhao}
\fancyhead[c]{Math 134 Homework \#5}
\fancyhead[r]{\today}
\fancyfoot[c]{\thepage}
\renewcommand{\headrulewidth}{0.2pt} %Creates a horizontal line underneath the header
\setlength{\headheight}{15pt} %Sets enough space for the header



\begin{document} %The writing for your homework should all come after this. 
%FromSection2.4: 4*,6,15*,19,23,36*. FromSection1.1: 3,5,7,12*,17,34*.

%Enumerate starts a list of problems so you can put each homework problem after each item. 
\begin{enumerate}[start=1,label={\bfseries Problem \arabic*:},leftmargin=1in] %You can change "Problem" to be whatever label you like. 
    \item 
    \begin{enumerate}
        \item $\dot{x} = v, \dot{v} = -\omega^{2}x$. 
        
        Dividing the two, we get 
        \[
            \frac{dx}{dt} \cdot \frac{dt}{dv} = \frac{dx}{dv} = \frac{v}{-\omega^{2}x}
        \]
        Doing some rearranging and integration we get 
        \[
            -\frac{\omega^{2}}{2}x^{2} = \frac{1}{2}v^{2} + c
        \]

        So, $\omega^{2}x^{2} + v^{2} C$.  

        \item We know that $E = \frac{1}{2}mv^{2} + \frac{1}{2}kx^{2}$. 
        Let $k = \omega^{2}$. Then, $\frac{1}{2}mv^{2} + \frac{1}{2}\omega^{2}x^{2} = E$. 
        
        Mulitplying by $2m$ which should be constant, $C = 2mE = v^{2} + \omega^{2}x^{2}$. 
    \end{enumerate}    

    \item We get that $A = \begin{bmatrix}
        0 & 1 \\ 
        -1 & 1
    \end{bmatrix}$. Solving for the eigenvalues of the vector, we find that 
    \[
        \lambda = \frac{1 \pm \sqrt{1 - 4(1-2)}}{2} = \frac{1}{2} \pm \frac{\sqrt{3}}{2}i
    \]
    Since the eigenvalue is a complex number, and the real part is negative, we find that the system is a unstable spiral. 
    To interpret this in terms of Romeo and Juliets love, we can say that Romeo and Juliets feelings towards each other get more and more intense as time progresses. 
    

    \item Given $\ddot{\theta} + \sin \theta = \gamma$. 
    Let $v = \dot{\theta}$. So $\dot{v} + \sin \theta = \gamma$. 

    We know that $\dot{(\frac{1}{2}v^2)} = v\dot{v}$. 

    Let's multiply the original equation by $v$. 
    \[ 
        v\dot{v} + v\sin\theta - \gamma v = 0
    \]
    Integrating, we get 
    \[ 
        \frac{1}{2}v^{2} + \sin \theta - \gamma \theta = C
    \]
    Since this only depends on $\theta$ and $v$, we find that it is a conservative system. 

    If we map $\theta \rightarrow \theta$ and $v \rightarrow -v$, we find that $\frac{d(-v)}{d(-t)} + \sin \theta = \gamma$. 
    This simplifies to $\frac{dv}{dt} + \sin \theta = \gamma $. Hence, the system is reversible on this map.
\end{enumerate}

\end{document}
