\documentclass[12pt]{article}

\usepackage{graphicx}			% Use this package to include images
\usepackage{amsmath}			% A library of many standard math expressions
\usepackage{amssymb}
\usepackage[margin=1in]{geometry}% Sets 1in margins. 
\usepackage{fancyhdr}			% Creates headers and footers
\usepackage{enumerate}          %These two package give custom labels to a list
\usepackage[shortlabels]{enumitem}


% Creates the header and footer. You can adjust the look and feel of these here.
\pagestyle{fancy}
\fancyhead[l]{Anthony Zhao}
\fancyhead[c]{Math 134 Homework \#6}
\fancyhead[r]{\today}
\fancyfoot[c]{\thepage}
\renewcommand{\headrulewidth}{0.2pt} %Creates a horizontal line underneath the header
\setlength{\headheight}{15pt} %Sets enough space for the header



\begin{document} %The writing for your homework should all come after this. 
%FromSection2.4: 4*,6,15*,19,23,36*. FromSection1.1: 3,5,7,12*,17,34*.

%Enumerate starts a list of problems so you can put each homework problem after each item. 
\begin{enumerate}[start=1,label={\bfseries Problem \arabic*:},leftmargin=1in] %You can change "Problem" to be whatever label you like. 
    \item $(6.8.1)$
    stable spiral = 1

    \includegraphics[scale=0.6]{/Users/anthonyzhao/Desktop/UCLA-Math/Math134/images/Screenshot 2025-02-28 at 9.47.27 PM.png}
    
    unstable spiral = 1

    \includegraphics[scale=0.3]{/Users/anthonyzhao/Desktop/UCLA-Math/Math134/images/Screenshot 2025-02-28 at 9.51.21 PM.png}

    center = 1

    \includegraphics[scale=0.3]{/Users/anthonyzhao/Desktop/UCLA-Math/Math134/images/Screenshot 2025-02-28 at 9.54.05 PM.png}

    star = 1

    \includegraphics[scale=0.3]{/Users/anthonyzhao/Desktop/UCLA-Math/Math134/images/Screenshot 2025-02-28 at 9.56.11 PM.png}

    degenerate node = 1
    \includegraphics[scale=0.3]{/Users/anthonyzhao/Desktop/UCLA-Math/Math134/images/Screenshot 2025-02-28 at 9.58.29 PM.png}

    \item $(6.8.6)$ In a closed orbit, $I_{\Gamma} = 1$. We also know that $I_{\Gamma} = I_{1} + I_{2} + \dots + I_{n}$
    Each saddle node contributes $-1$ to the index, while degenerate nodes, spirals, and centers all contribute $+1$ per. 
    Hence, we get $I_{\Gamma} = 1 = N + F + C - S$. 
    \item $(6.8.8)$ 
    \includegraphics[scale=0.3]{/Users/anthonyzhao/Desktop/UCLA-Math/Math134/images/Screenshot 2025-02-28 at 10.03.18 PM.png}

    We Know that $I_{C_{3}} = 1 = I_{C_{1}} + I_{C_{2}} + \text{index of any fixed points}$. Since, $I_{C_{1}} = I_{C_{2}} = 1$ because they are closed orbits, 
    there must be at least one fixed point that contributes -1 to the index of $I_{C_{3}}$ to fufill the first equality. 
    \item $(6.8.13)$
    \begin{enumerate}
        \item Let $u = \frac{g(x,y)}{f(x,y)}$.
        So, $du = \frac{(f)dg - (g)df}{f^{2}}$ 

        So, $\phi = \arctan(u)$ and $d\phi = \frac{du}{1 + u^{2}}$. 
        Substituting $du$, we get 
         \[
         d\phi = \frac{(f)dg - (g)df}{f^{2}(1 + \frac{g^{2}}{f^{2}})} 
         \]

         Simplifying some things, 
         \[ 
            d\phi = \frac{(f)dg - (g)df}{f^{2}(1 + \frac{g^{2}}{f^{2}})} = \frac{fdg - gdf}{f^{2}+g^{2}}
         \]

         \item So, $I_{C} = \frac{1}{2\pi} \oint d\phi = \frac{1}{2\pi} \oint \frac{fdg - gdf}{f^{2}+g^{2}}$ 
    \end{enumerate} 
    
    \item $(6.8.14bc)$
    $(b)$ 
    Let $x = r\cos \theta$ and $y = r\sin \theta$. Then 
    \begin{align*}
        f(x, y) = f(r, \theta) = r\cos\theta\cos \alpha -r\sin\theta \sin \alpha = r\cos(\theta + \alpha)\\
        g(x, y) = g(r, \theta) = r\cos\theta\sin \alpha +r\sin\theta \cos\alpha  = r\sin(\theta + \alpha )
    \end{align*}
    So, $\phi = \arctan(\frac{r\sin(\theta + \alpha)}{r\cos(\theta + \alpha)}) = \arctan(\tan(\theta + \alpha)) = \theta + \alpha$. Thus, $d\phi = d\theta$. 

    Therefore, $I_{C} = \frac{1}{2\pi} \oint d\phi = \frac{1}{2\pi} \oint d\theta$. The integral is independent of $\alpha$. 
    
    
    $(c)$ 
    \[
        I_{C} = \frac{1}{2\pi} \oint d\theta = \frac{1}{2\pi} 2\pi = 1
     \]
\end{enumerate}

\end{document}
