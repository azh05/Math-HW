\documentclass[12pt]{article}

\usepackage{graphicx}			% Use this package to include images
\usepackage{amsmath}			% A library of many standard math expressions
\usepackage{amssymb}
\usepackage[margin=1in]{geometry}% Sets 1in margins. 
\usepackage{fancyhdr}			% Creates headers and footers
\usepackage{enumerate}          %These two package give custom labels to a list
\usepackage[shortlabels]{enumitem}


% Creates the header and footer. You can adjust the look and feel of these here.
\pagestyle{fancy}
\fancyhead[l]{Anthony Zhao}
\fancyhead[c]{Math 134 Homework \#2}
\fancyhead[r]{\today}
\fancyfoot[c]{\thepage}
\renewcommand{\headrulewidth}{0.2pt} %Creates a horizontal line underneath the header
\setlength{\headheight}{15pt} %Sets enough space for the header



\begin{document} %The writing for your homework should all come after this. 
%FromSection2.4: 4*,6,15*,19,23,36*. FromSection1.1: 3,5,7,12*,17,34*.

%Enumerate starts a list of problems so you can put each homework problem after each item. 
\begin{enumerate}[start=1,label={\bfseries Problem \arabic*:},leftmargin=1in] %You can change "Problem" to be whatever label you like. 
    \item 2.2.10 
    \begin{enumerate}
        \item $\dot{x} = f(x) = 0$. For all $x \in \mathbb{R}$, $\dot{x} = 0$, so every real number is a fixed point. 
        \item If there are no fixed points, for all $x \in \mathbb{R}$, $\dot{x} = f(x) \neq 0$. 
        One function that satisfies  $\dot{x} = 1$. 
    \end{enumerate}

    \item 2.4.1 
    
    $\frac{dx}{dt} = x(1-x)$. The fixed points are $x=0,1$.  

    \[
    \frac{df}{dt} = \frac{df}{dx} \cdot \frac{dx}{dt} = \frac{d}{dx}(x - x^{2}) \cdot 1 = 1 - 2x
    \]
    
    Evaluated at $x=0$, $\frac{df}{dt} = 1 > 0$. Hence, $x = 0$ is a unstable fixed point. 
    Evaluated at $x=1$, $\frac{df}{dx} = 1 - 2(1) = -1$. So, $x=1$ is a stable fixed point. 


    \item 2.4.2 
    
    $\dot{x} = x(1-x)(2-x)$. Fixed points: $x = 0, 1, 2$.  

    \[
    \frac{df}{dt} = 3x^{2} - 6x + 2
    \]

    Evaluated at $x=0$, $\frac{df}{dt} = 2 > 0$. So, $x=0$ is a unstable fixed point. 
    Evaluated at $x=1$, $\frac{df}{dt} = -1 < 0$. So $x=1$ is a stable fixed point. 
    Evaluated at $x=2$, $\frac{df}{dt} = -2 < 0$. So $x=2$ is a stable fixed point. 

    \item 2.4.4 
    
    $\dot{x} = x^{2}(6 - x)$. Fixed points: $x = 0, 6$. 

    \[
    \frac{df}{dt} = 12x - 3x^{2} 
    \]

    Evaluated at $x=0$, $\frac{df}{dt} = 0$. Linear Stability Analysis fails at $x=0$. 
    Evaluated at $x=6$, $\frac{df}{dt} = -36 < 0$. So $x=6$ is a stable fixed point. 

    \item 2.7.2 
    
    \begin{align*}
        -\frac{dV}{dx} &= 3 \\ 
        V &= -3x + c
    \end{align*}

    Choose $c = 0$. $V(x) = 0$ is not a local minimum and maximum, so equilibrium does not occur.
    This is obvious since $\dot{x} = 3 \neq 0$. 
\end{enumerate}

\end{document}
