\documentclass[12pt]{article}

\usepackage{graphicx}			% Use this package to include images
\usepackage{amsmath}			% A library of many standard math expressions
\usepackage{amssymb}
\usepackage[margin=1in]{geometry}% Sets 1in margins. 
\usepackage{fancyhdr}			% Creates headers and footers
\usepackage{enumerate}          %These two package give custom labels to a list
\usepackage[shortlabels]{enumitem}


% Creates the header and footer. You can adjust the look and feel of these here.
\pagestyle{fancy}
\fancyhead[l]{Anthony Zhao}
\fancyhead[c]{Math 134 Homework \#4}
\fancyhead[r]{\today}
\fancyfoot[c]{\thepage}
\renewcommand{\headrulewidth}{0.2pt} %Creates a horizontal line underneath the header
\setlength{\headheight}{15pt} %Sets enough space for the header



\begin{document} %The writing for your homework should all come after this. 
%FromSection2.4: 4*,6,15*,19,23,36*. FromSection1.1: 3,5,7,12*,17,34*.

%Enumerate starts a list of problems so you can put each homework problem after each item. 
\begin{enumerate}[start=1,label={\bfseries Problem \arabic*:},leftmargin=1in] %You can change "Problem" to be whatever label you like. 
    \item (3.4.8) 
    \begin{enumerate}
        \item $\frac{dx}{dt} = rx - \frac{x}{1+x^{2}} = 0$
    \begin{align*}
        x(r - \frac{1}{x^{2}+1}) &= 0\\
        x = 0, \quad r &= \frac{1}{x^{2}+1} \\
        \frac{df}{dt} = r + \frac{x^{2} - 1}{(x^{2} + 1)^{2}} &= 0\\
    \end{align*}
    Plugging in $x = 0$, we get $r = 1$. Thus, the only equilibrium point is $(0,1)$. We get a similar result when plugging in $r = \frac{1}{x^{2}+1}$. 

        \item Graphically, we can see that if $r < 1$ then there are two unstable equilibrium points and one stable equilibrium point at 0. 
        If $r > 1$, only the $x = 0$ equilibrium point remains, and it is unstable. Hence, we have a subcritical pitchfork bifurcation at $r = 1$.

        \includegraphics[scale=0.5]{/Users/anthonyzhao/Desktop/UCLA-Math/Math134/images/Screenshot 2025-02-04 at 3.26.29 PM.png}
    \end{enumerate}
    
    \item (4.4.1) $b\dot{\theta} + mgL\sin\theta = \tau - k\theta$
    
    \begin{align*}
       f(\theta) = \dot{\theta} &= \frac{\tau - k\theta - mgL\sin\theta}{b}\\
    \end{align*}

    If this was well defined on a circle, then $f(\theta) = f(\theta + 2\pi)$. However, because of the $k\theta$ term they are clearly different for all $k \neq 0$. 
Hence this vector field is only well defined on a circle if $k = 0$.  

    \item \begin{enumerate}
        \item $A = \begin{bmatrix}
            4 & -1 \\ 
            2 & 1 
        \end{bmatrix}$ and $x = \begin{bmatrix}
            x \\ y
        \end{bmatrix}$ 

        Indeed, the characteristic polynomial is $\lambda^{2} - 5\lambda + 6 = 0$ which has roots $\lambda = 2, 3$.

        The eigenvector with eigenvalue 2 is $\begin{bmatrix}
            1 \\ 2 
        \end{bmatrix}$ and the eigenvector with eigenvalue 3 is $\begin{bmatrix}
            1 \\ 1 \end{bmatrix}$.

        \item Plotting the graph reveals a spiral. The vectors are pointing away from the origin, hence, the origin is a unstable spiral.
    \end{enumerate}
\end{enumerate}

\end{document}
