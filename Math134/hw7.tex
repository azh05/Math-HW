\documentclass[12pt]{article}

\usepackage{graphicx}			% Use this package to include images
\usepackage{amsmath}			% A library of many standard math expressions
\usepackage{amssymb}
\usepackage[margin=1in]{geometry}% Sets 1in margins. 
\usepackage{fancyhdr}			% Creates headers and footers
\usepackage{enumerate}          %These two package give custom labels to a list
\usepackage[shortlabels]{enumitem}


% Creates the header and footer. You can adjust the look and feel of these here.
\pagestyle{fancy}
\fancyhead[l]{Anthony Zhao}
\fancyhead[c]{Math 134 Homework \#7}
\fancyhead[r]{\today}
\fancyfoot[c]{\thepage}
\renewcommand{\headrulewidth}{0.2pt} %Creates a horizontal line underneath the header
\setlength{\headheight}{15pt} %Sets enough space for the header



\begin{document} %The writing for your homework should all come after this. 
%FromSection2.4: 4*,6,15*,19,23,36*. FromSection1.1: 3,5,7,12*,17,34*.

%Enumerate starts a list of problems so you can put each homework problem after each item. 
\begin{enumerate}[start=1,label={\bfseries Problem \arabic*:},leftmargin=1in] %You can change "Problem" to be whatever label you like. 
    \item $(7.1.5)$ Note that $x^2 + y^2 = r^2$. For $x$, \begin{align*}
        \dot{x} &= \dot{r}\cos\theta - r\sin\theta \dot{\theta}\\
            &= r(1-r^2)\cos\theta -r \sin \theta \\ 
            &= (1-r^2)x - y \\
            &= x - y - x(x^2 + y^2)\\ 
    \end{align*}
    
    Similarly, expanding $\dot{y}$, 
    \begin{align*}
        \dot{y} &= \dot{r} \sin\theta + r\cos \theta \dot{\theta} \\ 
        &= r(1-r^2)\sin\theta + r\cos\theta \\ 
        &= y(1-r^2) + x \\
        &= x + y - y(x^2 + y^2)
    \end{align*}

    \item $(7.2.10)$ Take $\dot{V}$. 
    
    \begin{align*}
        \dot{V} = 2ax\dot{x} &= 2by\dot{y} \\ 
        &= 2ax(y - x^3) + 2by(-x-y^3) \\ 
        &= 2axy - 2ax^4 - 2bxy - 2b^4 \\ 
        &= (2a - 2b)xy - 2ax^4 - 2by^4\\ 
     \end{align*}
     Notice that if $2a - 2b = 0$ and $a, b > 0$, then the rest of the terms would be strictly negative for all 
     $x, y \in \mathbb{R}$ except the fixed point. Let $a = b = 1$. Then $V = x^2 + y^2 > 0$ for all $(x, y) \neq (0, 0)$, and $\dot{V} = -2x^4 - 2y^4 < 0$.  
     Therefore, since we can construct a Liapunov function, there cannot exist a closed orbit.

     \item $(7.2.12)$ let $V = x^m + ay^n$. 
     Then, 
     \begin{align*}
        \dot{V} &= mx^{m-1}\dot{x} + any^{n-1}\dot{y}\\ 
        &= -mx^m + (2my^3 - 2my^4) x^{m-1} - anxy^{n-1} + (an + axn)y^n \\ 
     \end{align*}
     Let $n = 4, m = 2, a = 1$. Then, 
     \[
     = -2x^2 - 4y^2 < 0
     \]
     for all $(x, y) \neq (0, 0)$. Also note that $V = x^2 + y^4 > 0$ for all $(x ,y) \neq (0, 0)$. Hence, since we can construct a Liapunov function, there cannot exist a closed orbit.

     \item $(7.2.18)$ By Dulac's Criterion, we want to show that $\nabla \cdot (g\dot{x})$ is one sign on the first quadrant.  
     \begin{align*}
        \nabla \cdot (g\dot{x}) &=  \frac{d}{dx} (g\dot{x}) + \frac{d}{dy}(g\dot{y}) \\ 
        &= y^{\alpha - 1} (-rx + \frac{r}{2} + \alpha - \frac{\alpha}{x})\\ 
     \end{align*}

     For $x, y \geq 0$, $r > 0$, we have $y^{\alpha-1} > 0$. 
     So we want $-rx + \frac{r}{2} + \alpha - \frac{\alpha}{x} < 0$. 

     According to ChatGPT (Reasoning Model), if we set $\alpha = \frac{r}{2}(1 - \frac{1}{\sqrt{2}})^2$ then $-rx + \frac{r}{2} + \alpha - \frac{\alpha}{x} < 0$ for all $x$. 
     Thus, by Dulac's Criterion, we cannot have a closed orbit in the first quadrant\dots
     
     \item $(8.2.12)$ 
     $(a)$
     Note that $f(x, y) = xy^2, g(x, y) = -x^2, \omega = 1$. Note the following derivations of $f$ and $g$.   
     \begin{align*}
        f_{xxx} &= 0\\
        f_{xyy} &= 2\\ 
        f_{xy} &= 2y\\ 
        f_{xx} &= 0\\ 
        f_{yy} &= 2x\\ 
        g_{xxy} &= 0\\ 
        g_{yyy} &= 0\\ 
        g_{xy} &= 0 \\
        g_{xx} &= -2\\
        g_{yy} &= 0
      \end{align*}

      So, $16a = 2 + 4xy$. Evaluated at $(0, 0)$, $a = \frac{2}{16}$. 
      
    $(b)$ This is greater than 0, so this is a subcritical Hopf bifurcation.
    \end{enumerate}
\end{document}
