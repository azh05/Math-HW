\documentclass[12pt]{article}

\usepackage{graphicx}			% Use this package to include images
\usepackage{amsmath}			% A library of many standard math expressions
\usepackage{amssymb}
\usepackage[margin=1in]{geometry}% Sets 1in margins. 
\usepackage{fancyhdr}			% Creates headers and footers
\usepackage{enumerate}          %These two package give custom labels to a list
\usepackage[shortlabels]{enumitem}


% Creates the header and footer. You can adjust the look and feel of these here.
\pagestyle{fancy}
\fancyhead[l]{Anthony Zhao}
\fancyhead[c]{Math 134 Homework \#3}
\fancyhead[r]{\today}
\fancyfoot[c]{\thepage}
\renewcommand{\headrulewidth}{0.2pt} %Creates a horizontal line underneath the header
\setlength{\headheight}{15pt} %Sets enough space for the header



\begin{document} %The writing for your homework should all come after this. 
%FromSection2.4: 4*,6,15*,19,23,36*. FromSection1.1: 3,5,7,12*,17,34*.

%Enumerate starts a list of problems so you can put each homework problem after each item. 
\begin{enumerate}[start=1,label={\bfseries Problem \arabic*:},leftmargin=1in] %You can change "Problem" to be whatever label you like. 
    \item $(2.4.7)$ $\dot{x} = ax-x^{3}$. Note that $\frac{df}{dx} = a - 3x^{2}$. 
    
    \textbf{Case:} $a=0$. Then, $\dot{x} = x^{3}$ = 0. So the system has one fixed point at $x = 0$. 
    $\frac{df}{dx} = 0-0 = 0$. So linear stability analysis fails. On a graphical inspection, $x=0$ is a stable fixed point. 

    \textbf{Case:} $a < 0$. The system has one fixed point at $x = 0$. 
    At $x = 0$, $\frac{df}{dx} = a < 0$, which shows that it is stable. 

    \textbf{Case:} $a > 0$. The system has three fixed points at $x = 0, \sqrt{a}, -\sqrt{a}$. 
    At $x = 0$, $\frac{df}{dx} = a > 0$, so it is unstable. 

    At $x = \sqrt{a}$, $\frac{df}{dx} = a - 3(a) = -2a < 0$. So it is stable. 

    At $x = -\sqrt{a}$, $\frac{df}{dx} = a - 3(a) = -2a < 0$. So it is stable as well.

    \item (3.2.1) $\dot{x} = rx + x^{2}$ 
    
    \includegraphics[scale = 0.47]{/Users/anthonyzhao/Desktop/UCLA-Math/Math134/images/Screenshot 2025-01-26 at 6.10.59 PM.png}

    \begin{align*}
        \frac{df}{dx} = r + 2x = 0\\ 
        r= -2x \\ 
        \frac{dx}{dt} = rx + x^{2} = x(r+x) = 0
    \end{align*}
    So, $x = -r$  and $x = 0$ are solutions of $\frac{dx}{dt} = 0$. Plugging in both, results in $r_{c} = 0$. 

    \includegraphics[scale=0.5]{~/Desktop/UCLA-Math/Math134/images/Screenshot 2025-01-26 at 6.22.52 PM.png}
    \item $(3.4.5)$ $\dot{x} = r - 3x^{2}$. 
    
    Using a change of variables of $x = -\frac{1}{3}u$, we get $\frac{du}{dt} = -3r + u^2$, so this is a saddle node bifurcation.
    
    \begin{align*}
        \frac{dx}{dt} = r - 3x^{2} = 0 \\ 
            r = 3x^{2}\\
        \frac{df}{dx} = 6x = 0 \\ 
        r_{c} = 0
    \end{align*}

    \includegraphics[scale=0.75]{/Users/anthonyzhao/Desktop/UCLA-Math/Math134/images/Screenshot 2025-01-26 at 6.35.45 PM.png}
\end{enumerate}

\end{document}
